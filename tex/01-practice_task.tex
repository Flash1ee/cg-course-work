\thispagestyle{empty}
\begin{center}
    \fontsize{11pt}{0.3\baselineskip}\selectfont \textbf{Министерство науки и высшего образования Российской Федерации \\ Федеральное государственное бюджетное образовательное учреждение \\ высшего образования \\ <<Московский государственный технический университет имени Н.Э. Баумана \\ (национальный исследовательский университет)>> \\ (МГТУ им. Н.Э. Баумана)}
    % \fontsize{1pt}{0.3\baselineskip}\selectfont
    \makebox[\linewidth]{\rule{\textwidth}{3pt}}
    \makebox[\linewidth]{\rule{\textwidth}{3pt}}
    
    \vspace{\baselineskip}

    \fontsize{12pt}{\baselineskip}\selectfont 

    Кафедра << \uline{Программное обеспечение ЭВМ и информационные технологии} >> ( \uline{ИУ7} )
\end{center}


\begin{center}
    \fontsize{18pt}{\baselineskip}\selectfont \textbf{З А Д А Н И Е}\\
    \fontsize{16pt}{\baselineskip}\selectfont \textbf{на прохождение производственной практики}
\end{center}

\normalsize

\begingroup
\fontsize{12pt}{1\baselineskip}\selectfont
% \setlength{\parskip}{0.1em}
\setlength{\parindent}{0em}
на предприятии \uline{\hfill МГТУ им. Н. Э. Баумана, каф. ИУ7 \hfill}

Студент \uline{\hfill Варин Дмитрий Владимирович ИУ7-46Б \hfill}

Во время прохождения производственной практики студент должен:

1. Спроектировать программу для моделирования твердых тел с помощью примитивов (куб, сфера) и логических операций (пересечение, объединение, разность).  

2. Проанализировать методы и алгоритмы. Выбрать необходимые для решения поставленной задачи.

3. Разработать архитектуру приложения. 

\vfill

Дата выдачи задания
 <<\uline{\mbox{\hspace*{5mm}}}>> \uline{\mbox{\hspace*{2.5cm}}} 20\uline{21} г.

\endgroup

\vfill

\begin{table}[h!]
    \fontsize{12pt}{0.7\baselineskip}\selectfont
    \centering
 

    \begin{signstabular}[0.7]{p{7.25cm} >{\centering\arraybackslash}p{4cm} >{\centering\arraybackslash}p{4cm}}
        Руководитель практики от кафедры & \uline{\mbox{\hspace*{4cm}}} & \uline{\hfill Куров А.В. \hfill} \\
        & \scriptsize (Подпись, дата) & \scriptsize (Фамилия И.О.)
    \end{signstabular}
    \vspace{\baselineskip}

    \begin{signstabular}[0.7]{p{7.25cm} >{\centering\arraybackslash}p{4cm} >{\centering\arraybackslash}p{4cm}}
        Студент & \uline{\mbox{\hspace*{4cm}}} & \uline{\hfill Варин Д. В. \hfill} \\
        & \scriptsize (Подпись, дата) & \scriptsize (Фамилия И.О.)
    \end{signstabular}

    \vspace{\baselineskip}
\end{table}


% \begin{flushleft}
%     \fontsize{11pt}{0.5\baselineskip}\selectfont
%     \uline{Примечание:} Задание оформляется в двух экземплярах -- один выдается студенту, второй хранится на кафедре
% \end{flushleft}

\clearpage
\thispagestyle{empty}

\begin{center}
    \fontsize{12pt}{\baselineskip}\selectfont
    \textit{Дополнительные указания по проектированию}
\end{center}

\begingroup
\fontsize{12pt}{0.7\baselineskip}\selectfont
\setlength{\parskip}{0em}
\setlength{\parindent}{0em}

\uline{\mbox{\hspace*{1.25cm}} Пользователь должен иметь возможность 
комбинировать тела с помощью операций композиции (объединение, пересечение, разность).
Модель должна состоять только из заданных в программе примитивов и их комбинаций.
На сцене должна присутствовать 1 сконструированная модель.
На сцене должна присутствовать одна камера.
Пользователь должен иметь возможность выполнять следующие действия: 
использовать операции композиции для создания модели,
использовать операции трансформации модели,
выбирать примитивы, используемые при моделировании.
Модель располагается в центре сцены, пользователь может изменять положение модели посредством операций трансформации (перемещение, масштабирование, поворот).
    \hfill
}
\endgroup
\normalsize