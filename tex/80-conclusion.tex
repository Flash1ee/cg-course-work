\Conclusion % заключение к отчёту
 
Во время выполнения курсового проекта было реализованно приложение для моделирования твердых тел
на основе примитивов (куб, сфера, цилиндр) с помощью
логических операций (объединение, пересечение, разность). Приложение работает в режиме реального времени.
Были проанализированы и рассмотрены существующие методы создания моделей и их отрисовки.
В качестве метода создания был выбран метод Конструктивной сплошной геометрии.
В качестве метода отрисовки - Raymarching с помощью видеокарты.
В ходе выполнения поставленной задачи были получены знания в области компьютерной графики,
закреплены навыки проектирования программного обеспечения, а поиск оптимальных решений для эффективной работы программного обеспечения позволил улучшить навыки анализа информации.
 
В результате проведенной работы было получено программное обеспечение,
позволяющие моделировать тела из примитивов.
В ходе выполнения эксперементально-исследовательской части было
установленно, что применимость программного обеспечения в режиме реального
времени возможно при использовании дискретной видеокарты, так как встроенная
не позволяет получить FPS, которых хватает для использования приложения. Было выявлено,
что отрисовки сцены на дискретной видеокарте в среднем позволяет получить в 4 раза больше
FPS в сравнении с встроенной. Было выявлено, что при конфигурации системы, на который производилось
тестирование оптимальное количество объектов на сцене находится в диапазоне от 1 до 7 объектов.
 
%%% Local Variables:
%%% mode: latex
%%% TeX-master: "rpz"
%%% End:
 
 

