\thispagestyle{empty}
\begin{center}
    \fontsize{10pt}{0.3\baselineskip}\selectfont \textbf{Министерство науки и высшего образования Российской Федерации \\ Федеральное государственное бюджетное образовательное учреждение \\ высшего образования \\ <<Московский государственный технический университет имени Н.Э. Баумана \\ (национальный исследовательский университет)>> \\ (МГТУ им. Н.Э. Баумана)}
    % \fontsize{1pt}{0.3\baselineskip}\selectfont
    \makebox[\linewidth]{\rule{\textwidth}{3pt}}
    \begin{flushright}
        \fontsize{11pt}{0.5\baselineskip}\selectfont
            УТВЕРЖДАЮ \\ Заведующий кафедрой \textbf{ИУ7}, \\
            \uline{\mbox{\hspace*{2cm}}}
            \textbf{И.В. Рудаков} \\
            <<\uline{\mbox{\hspace*{1cm}}}>>
            \uline{\mbox{\hspace*{2.5cm}}}
            \textbf{2021} г.
    \end{flushright}
\end{center}


\begin{center}
    \fontsize{16pt}{0.5\baselineskip}\selectfont \textbf{З А Д А Н И Е}\\
    \fontsize{14pt}{0.5\baselineskip}\selectfont \textbf{на выполнение курсовой работы}
\end{center}

\normalsize

\begingroup
\fontsize{11pt}{0.5\baselineskip}\selectfont
\setlength{\parskip}{0.1em}
\setlength{\parindent}{0em}
по дисциплине \uline{\hfill Компьютерная графика \hfill}

Студент группы \uline{\hfill ИУ7-56Б \hfill}

\uline{\hfill Варин Дмитрий Владимирович \hfill}

Тема курсовой работы \uline{Разработка программного обеспечения для моделирования твёрдых тел на основе примитивов и логических операций.\hfill}

Направленность КР (учебная, исследовательская, практическая, производственная, др.)

\uline{\hfill учебная \hfill}

Источник тематики (кафедра, предприятие, НИР) \uline{\hfill кафедра \hfill}

График выполнения работы:  25\% к \uline{4} нед., 50\% к \uline{7} нед., 75\% к \uline{11} нед., 100\% к \uline{14} нед.

\textbf{\textit{Задание}}
\uline{Разработать программу для моделирования твердых тел с помощью примитивов (куб, сфера, цилиндр) и логических операций (пересечение, объединение, разность). 
Предоставить пользователю возможность выбирать примитивы, из которых моделируется тело, операции композиции тел и трансформации. 
Модель должна состоять только из заданных в программе примитивов.
На сцене должна присутствовать одна камера, положение которой зафиксировано.
Пользователь должен иметь возможность выполнять следующие действия: 
использовать операции композиции для создания модели,
использовать операции трансформации модели,
выбирать примитивы, используемые при моделировании.
Модель располагается в центре сцены, пользователь может изменять положение посредством операций трансформации.
    \hfill}

\textbf{\textit{Оформление курсовой работы:}}

Расчетно-пояснительная записка на 25-30  листах формата А4.

\uline{Расчетно-пояснительная записка должна содержать постановку введение, аналитическую часть, конструкторскую часть, технологическую часть, экспериментально-исследовательский раздел, заключение, список литературы, приложения.
    \hfill}

Перечень графического материала (плакаты, схемы, чертежи и т.п.)

\uline{На защиту проекта должна быть представлена презентация, состоящая из 15-20 слайдов. На слайдах должны быть отражены: постановка задачи, использованные методы и алгоритмы, расчетные соотношения, структура комплекса программ, диаграмма классов, интерфейс, характеристики разработанного ПО, результаты проведенных исследований.
    \hfill}

Дата выдачи задания
 <<\uline{\mbox{\hspace*{5mm}}}>> \uline{\mbox{\hspace*{2.5cm}}} 20\uline{21} г.

\endgroup

% \vfill

\begin{table}[h!]
    \fontsize{11pt}{0.7\baselineskip}\selectfont
    \centering
    \begin{signstabular}[0.7]{p{7.25cm} >{\centering\arraybackslash}p{4cm} >{\centering\arraybackslash}p{4cm}}
        Студент группы ИУ7-56Б & \uline{\mbox{\hspace*{4cm}}} & \uline{\hfill Д. В. Варин  \hfill} \\
        & \scriptsize (Подпись, дата) & \scriptsize (И.О. Фамилия)
    \end{signstabular}

    % \vspace{\baselineskip}

    \begin{signstabular}[0.7]{p{7.25cm} >{\centering\arraybackslash}p{4cm} >{\centering\arraybackslash}p{4cm}}
        Руководитель курсовой работы & \uline{\mbox{\hspace*{4cm}}} & \uline{\hfill А.А. Волкова \hfill} \\
        & \scriptsize (Подпись, дата) & \scriptsize (И.О. Фамилия)
    \end{signstabular}
    \vspace{\baselineskip}
\end{table}


% \end{flushleft}

% \clearpage
% \thispagestyle{empty}

% \begin{center}
%     \fontsize{12pt}{\baselineskip}\selectfont
%     \textit{Дополнительные указания по проектированию}
% \end{center}

% \begingroup
% \fontsize{12pt}{0.7\baselineskip}\selectfont
% \setlength{\parskip}{0em}
% \setlength{\parindent}{0em}

% \uline{\mbox{\hspace*{1.25cm}} Пользователь должен иметь возможность 
% выбирать примитивы (куб, сфера), операции композиции (объединение, пересечение, разность).
% Модель должна состоять только из заданных в программе примитивов.
% На сцене должна присутствовать 1 сконструированная модель.
% На сцене должна присутствовать одна камера, положение которой зафиксировано.
% Пользователь должен иметь возможность выполнять следующие действия: 
% использовать операции композиции для создания модели,
% использовать операции трансформации модели,
% выбирать примитивы, используемые при моделировании.
% Модель располагается в центре сцены, пользователь может изменять положение посредством операций трансофрмации.
%     \hfill
% }


% % \uline{\mbox{\hspace*{1.25cm}} Пользователь должен иметь возможность сбросить программу в начальное состояние. Пользователь должен иметь возможность изменять примитивы, использующиеся в моделировании конкретной модели .\hfill}



% \endgroup
\normalsize
