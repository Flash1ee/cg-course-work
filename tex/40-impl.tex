\chapter{Технологический раздел}
\label{cha:impl}
В данном разделе представленны средства разработки программного обеспечения, детали реализации и тестирование функций.
\section{Средства реализации}
В качестве языка программирования, на котором будет реализовано программное обеспечение, выбран язык программирования JavaScript \cite{impl:js}. Выбор языка обусловлен тем, что данный язык является языком программирования для бразуера,
что позволяет запускать приложение в браузере и делает его кроссплатформенным решением, а также позволяет запускать приложение без установки дополнительных зависимостей. Помимо этого, для JavaScript существует бибиотека 
ThreeJS \cite{impl:three_js}, которая предоставляет canvas (холст), на котором происходит отрисовка сцены, модуль для работы с камерой, а также позволяет подключить шейдеры, используя небольшое количество подготовительных этапов.
Для создания пользовательского интерфейса программного обеспечения будет использоваться модуль dat-gui \cite{impl:dat_gui}. Этот модуль позволяет предоставить пользователю возможность изменять параметры модели.
Для мониторинга производительности будет использоваться модуль Stats \cite{impl:stats_js}. Этот модуль позволяет отслеживать FPS (количество кадров в секунду). Данная информация поволит оценить производительность ПО.
Функциональное тестирование ПО проводиться не будет из-за своей специфики -- разработанное ПО является GUI-приложением, что усложняет процесс тестирования.
В качестве среды разработки выбран текстовый редактор Visual Studio Code \cite{impl:vscode}, содержащий большое количеством плагинов и инструментов для различных языков программирования, в том числе JavaScript. Такие инструменты облегчают и ускоряют процесс разработки программного обеспеченияd.
Для запуска приложения используется python сервер \cite{impl:python}, позволяющий использовать ПО в браузере.  
\section{Детали реализации}


%%% mode: latex
%%% TeX-master: "rpz"
%%% End:
