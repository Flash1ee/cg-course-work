\Introduction
Сегодня для увеличения эффективности труда и повышения качества разрабатываемой продукции двухмерное проекционное черчение заменяется трехмерным (твердотельным) моделированием, которое работает с объектами, состоящими из замкнутого контура. 
Данный подход обеспечивает полное описание трехмерной геометрической формы \cite{sapr}.

Моделирование твердого тела является неотъемлемой частью проектирования и разработки изделий \cite{article:solid_modeling}. 
Все тела можно разделить на базовые и составные. К базовым относятся примитивы: параллелепипед, цилиндр, шар, конус и др..
Однако в жизни редко можно встретить объекты, состоящие из одного базового тела. Как правило, изделия сложны по своей структуре, что приводит к появлению составных.
Такие тела формируются в результате операций над базовыми (булевы функции сложения, вычитания, пересечения) \cite{sapr}.
Существует несколько схем представления таких тел, из которых нужно выбрать наиболее подходящую.


Смоделировав тело, предстаёт новая задача: нужно его визуализировать, а также предусмотреть возможность просмотра модели с разных ракурсов. Для этого есть несколько способов, каждый из которых имеет свои преимущества и недостатки \cite{article:3d}. 
Необходимо выбрать наиболее подходящий под выделенную задачу. 


После создания тела нужно его отрисовать. Это требует больших вычислительных мощностей. 
Данную задачу можно возложить на центральный процессор - CPU или графический - GPU \cite{cpu_or_gpu}. Следует выбрать наиболее подходящий вычислительный ресурс и отрендерить созданную модель. 
Все эти действия приводят к идее создания программного обеспечения, которое объединит в себе решение всех озвученных выше задач и приведёт к конечному результату - твердотельной модели.

Цель работы на время практики - проектирование программного обеспечения для моделирования твердых тел на основе примитивов и логических операций. 
Таким образом, необходимо выбрать оптимальные алгоритмы представления твердотельной модели, её преобразования, 
визуализации и аппаратной обработки. 
Спроектировать процесс моделирования и предоставить схему для его реализации.
% \begin{itemize}
% \item проанализировать методы создания сложных моделей;
% \item проанализировать методы обработки и рендера моделей;
% \item реализовать алгоритмы для создания поверхностей и их рендера с возможностью изменения ракурса обзора посредством камеры;
% % \item спроектировать и реализовать метод создания поверхностей;
% % \item спроектировать и реализовать метод рендера поверхностей;
% % \item реализовать возможность взаимодействия с моделью посредством камеры;
% \item проверить работоспособность ПО.
% \end{itemize}

\Abbrev{ПО}{"" программное обеспечение}
\Abbrev{CSG}{Constructive solid geometry ""--- конструктивная сплошная геометрия}
\Abbrev{SDF}{Signed distance functions ""--- поле расстояния с знаком}
\Abbrev{CPU}{Central processing unit ""--- центральный процессор}
\Abbrev{GPU}{Graphics processing unit ""--- графический процессор}
\Define{Raymarching}{алгоритм трассировки лучей}



